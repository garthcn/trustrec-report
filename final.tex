
\documentclass[12pt]{article}
\usepackage{geometry} % see geometry.pdf on how to lay out the page. There's lots.
\geometry{letterpaper} % or letter or a5paper or ... etc
% \geometry{landscape} % rotated page geometry

% See the ``Article customise'' template for come common customisations

\title{Final Report}
\author{}
%\date{} % delete this line to display the current date

%%% BEGIN DOCUMENT
\begin{document}

\maketitle


\section{Background}
Background.
\section{Methodology}
Methodology.

\subsection{Identifying Experts} % (fold)
\label{sub:identifying_experts}

Given a category $C$ and a set of users $U = \{{u_{1}, ..., u_{n}}\}$ who have written at least one review in category $C$, we primarily consider two factors in calculating a user's expertise: (1) the average quality of the user's reviews, and (2) the total number of user's reviews. We use the following formulas to calculate a user's expertise in a category $C$:
\begin{equation}
E(u) = \frac{\sum_{i=1}^n R_{u, i}} {n} \times f(n)
\end{equation}
\begin{equation}
f(n) = \frac{1}{1 + e^{\frac{-10n} {MAX\_REVIEWS(C)}}}
\end{equation}
where $R_{u, i}$ is the rating for user $u$'s review $i$ in category $C$ and $f(n)$ is a sigmoid function to avoid favoring the number of reviews too much. It is based on the assumption that a user written 100 reviews has more expertise than a user written 10 while a user written written 1000 reviews is almost as good as a user written 900 reviews. $MAX\_REVIEWS(C)$ is the maximum number of reviews per user in category $C$. $f(n)$ is tailored for every category, i.e., the user who has written the most reviews in a category have $f(n)=1$.

% subsection identifying_experts (end)

\subsection{Top $k$ Experts Recommendation} % (fold)
\label{sub:expert_recommendation}

Previous research has demonstrated that recommending items to users based on \emph{expert} opinions can be as good as traditional user-based \emph{Collaborative Filtering} \cite{Amatriain:2009p101}. We propose a similar expert recommendation algorithm: given an item $i$ that belongs to a category $C$, the recommender first picks out the top $k$ percent experts in category $C$; then for each expert, if the expert has rated item $i$, the exact rating is returned; otherwise a weighted average of similar items of $i$ is returned; finally the recommender averages the ratings of all selected experts by their expertise as the predicted rating for item $i$. Item similarity is calculated using \emph{Pearson Correlation}:
\begin{equation}
	corr(i,j) = \frac{\sum_{u \in U_{i,j}} (r_{u,i} - \bar{r}_u)(r_{u,j} - \bar{r}_u)} {{\sqrt{\sum_{u \in U_{i,j}} (r_{u,i} - \bar{r}_u)}} \sqrt{\sum_{u \in U_{i,j}} (r_{u,j} - \bar{r}_u)}}
\end{equation}
where $U_{i,j}$ is the set of common users who have rated both items $i$ and $j$, and $\bar{r}_u$ denotes the average of ratings expressed by $u$.

% subsection expert_recommendation (end)

\section{Experiment Results}
Results.


\bibliographystyle{ieeetr}
\bibliography{references}


\end{document}

